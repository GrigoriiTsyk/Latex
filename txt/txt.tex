\documentclass{article}
\usepackage[utf8]{inputenc}
\usepackage[russian]{babel}
\usepackage[utf8]{inputenc}
\usepackage[T2A]{fontenc}
\usepackage[russian]{babel}
\usepackage{indentfirst}
\usepackage{graphicx}
\usepackage{hyperref}
\usepackage{etoolbox}
\usepackage{float}
\usepackage{wrapfig}
\usepackage{multicol}
\usepackage{multirow,tabularx}
\usepackage{listings}
\usepackage[table,dvipsnames]{xcolor}
\usepackage{soul}
\usepackage{amsmath,amsfonts,amssymb,amsthm}
\usepackage{fp}
\usepackage{array}
\usepackage{adjustbox}
\usepackage{geometry}
\setlength{\parindent}{1.25cm}

\newcommand{\RomanNumeralCaps}[1]
    {\MakeUppercase{\romannumeral #1}}

\begin{document}
\textbf{\S 2. Следящая система с люфтом.} Рассмотрим простейшую
следящую систему с люфтом в контактном устройстве и в зуб
чатом зацеплении, описываемую безразмерным уравнением [152]
\begin{gather}
\ddot{x}+\dot{x}=S(x, \dot{x})
\end{gather}
где $x$ - координата сервомотора и $S(x, \dot{x})$ - кусочно-постоянная
(характеризующая безразмерную э. д. с. и сухое трение в системе). 
Общеизвестным приемом при исследовании точечных преобразований 
является представление и исследование точечного преобразования в 
параметрической форме, где в качестве параметра вводится время пробега 
изображающей точки по траекториям системы между точками сшивания.%abzac

Особенностью рассматриваемой задачи является возможность
другого эффективного параметрического представления точечного
преобразования с введением в качестве параметров некоторых
отрезков в фазовом пространстве. Этот прием имеет значение, вы
ходящее за рамки рассматриваемой задачи.%abzac

Разбиение плоскости $(x, \dot{x})$ на области, где $C(x, \dot{x})$ сохраняет
постоянное значение, производится в зависимости от двух пара-
метров $k$ и $z$, характеризующих соответственно люфт в котакт
ном усростве и люфт в зацеплении.%abzac

Запишем уравнение (1) в виде системы
\begin{gather}
\dot{x}=y, \dot{y}=S(x,y)-y
\end{gather} 
и будем рассматривать фазовые траектории на плоскости $(x, y)$.
Разбиение фазовой плоскости на траектории будет симметрично
относительно начала координат, если за начало отсчета принять
середину максимального интерва
ла длиной $z+k$, который сервомо
тор может пройти по инерции. На
рис. 210 изображено разбиение
плоскости $(x, y)$ на десять об
ластей, где $S(x, y)$ сохраняет
постоянные значения, указан
ные на рисунке. Полосы ши
риной $y0$, примыкающие к оси $x$
сверзу или снизу, соответствуют
выбиранию сервомотором люфта
в зубчатом зацеплении, и для
них соответственно $S(x, y)=1$
или $S(x, y)=-1$ (сухим трением
при свободном движении сервомотора пренебрегаем). Полосе шириной $k$, содержащей внутри ось $y$, соответствует выбирание сер
вомотором совместно со следящей осью люфта в коптактном
устройстве при движении по инерции. Здесь $S(x, y)=-r$ или
$S(x, y)=r$ характеризует твердое трение в системе. На других
участках фазовой плоскости величина $S(x, y)$ имеет значение
$\pm 1 \pm r$, где знаки выбираются в зависимости от знака скорости
и знака включенной э. д. с. или 0, если люфт в зацеплении про
ходится по инерции. Величина $y_{0}$-максимаьная скорость, до
которой разгоняется сервомотор, выбирая люфт в зацеплении,
есть однозначная функция параметра $z$ и определяется урав
нением
\begin{gather}
z+y_{0}+ln(1-y_{0})=0.
\end{gather}
Это уравнение получатется, если в (2) положить $S(x,y)=1$ и
потребовать для решения системы (2) выполнения условий $x=$
$=-x_{0}, y=0; x=-x_{0}+z, y=y_{0}$.%abz

Построим точечное преобразование в себя полупрямой $L:$
$y=0, x\leq\frac{-(z+k)}{2}$, примыкающей слева к отрезку покоя: $y
=0, \frac{-(z+k)}{2}<x<\frac{z+k}{2}$. Так как фазовое пространство
симметрично оносительно начала координат, то задача сводится
к построению точечного отображнеия полупрямой $L$ в симметрич
ную полупрямую $L'$, примыкающую к отрезку покоя справа.%абзац

Рассмотрим траекторию в верзней полуплоскости, сшитую из
четырех ксков, начинающуюся в точке $(-u, 0)$ и заканчиваю
щуюся в точке $(v,0)$. ""Сшивание" траекторий в точках разры
ва правых частей системы совершается элементарно, если знак
правой части второго из уравнений (2) не изменяется при пере
ходе через линию сшивания. Так будет, если $y_{0}\leq1-r$, т. е. если
$r$ "не слишком велико". Точки пересечения этой траектории с по
лосой ширины $k$ будут $x=\frac{z-k}{2}, y=\eta$ и $x=\frac{z+k}{2}, y=x_{i}$. Как
оказывается, величины $\eta$ и $\xi$ целесообразно рассматривать как
параметры точечного преобразования.%абзац

Из уравнения (2), полагая $S(x, y)=1$ для первого куска
траектории и $S(x, y)=1-r$ для второго и используя условия
для концов кусков траекторий: $x=-u, y=0; x=-u+z, y=y_{0};$
$x=\frac{z-k}{2} y=r_{i}$, получим
\begin{gather}
u=\frac{z+k}{2} + (1-r)ln\frac{1-r-y_{0}}{1-r-\eta}+y_{0}-\eta, y_{0}\leq\eta<1-r.
\end{gather}
Полагая далее $S(x,y)=-r$ для третьего куска траектории и
$S(x,y)=-1-r$ для четвертого и используя условия для концов
кусков траекторий
\begin{center}
$x=\frac{z-k}{2}, y=\eta; x=\frac{z+k}{2}, y=\xi; x=v, y=0,$
\end{center}
получим
\begin{gather}
r ln(\xi + r) - r ln(\xi+r)+\eta-\xi-k=0,
\end{gather}
\begin{gather}
v=\frac{z+k}{2}+\xi+(1+r)ln\frac{1+r}{\xi+1+r}, 0\leq\xi<\infty. 
\end{gather}
Уравнения (4)-(6) определяют требуемое точечное преобра
зование в параметрической форме с двумя параметрами $\eta$ и $\xi$.
Разбиение фазового пространства $(x,y)$ на траектории определя
ется взаиморасположением кривых $u=u(\eta)$ и $v=v(\eta)$ на сов
мещенных плоскостях $(\nu, u)$ и $(\nu, v)$. Исследование взаиморас
положения кривых проводится элементарно при использовании
$\eta$ и $\xi$ как параметров.%абзац

Из (5) и(6) находим
\begin{center}
$\frac{d\eta}{d\xi}=\frac{\xi(\eta+r)}{\eta(\xi+r)}> 0, \frac{dv}{d\xi}=\frac{\xi}{\xi+1+r}>0.$
\end{center}
Откуда
\begin{gather}
\frac{dv}{d\eta}=\frac{\eta}{1+r+\xi}\frac{\xi+r}{\eta+r}>0.
\end{gather}
Из (4) имеем\\%абзац
\begin{gather}
\frac{du}{d\eta}=\frac{\eta}{1-r-\epsilon}.
\end{gather}
Сравнивая (7) и (8), непосредственно обнаруживаем, что для
любого $\eta$ будет
\begin{center}
$du/d\eta>dv/d\eta$,
\end{center}
и, следовательно, если существует точка пересечения кривых
$u=u(\eta)$ и $v=v(\eta)$, то она единственная и соответствует устойчивой,
неподвижной точке преобразования.

Граничные значения кривых $u=u(\eta)$ и $v=v(\eta)$ будут
\begin{center}
$u=(z+k)/2, \space{0,2cm} v=(z+k)/2$
\end{center}
соответственно при значениях параметров $\eta=y_{0}$и $\eta=y_{1}$ ($y_{1}$ определяется 
как корень уравнения (5) при $\xi=0$).

Для значений $\eta$, близких к $1-r$ ($ \eta=1-r$ - асимптота для
$u=u(\eta)$), будет $u>v$. Точка пересечения кривых $u=u(\eta)$ и
$v=v(\eta)$, будет поэтому существовать,
если $y_{1}<y_{0}$. Граница области существования
неподвижной точки преобразования
и соответствующего ей устойчивого
предельного цикла определяется
условием $y_{1}=y_{0}$.

Уравнение (3) совместно с уравнением
\begin{gather}
r ln r-r ln(y_{0}+r)+y_{0}-k=0,
\end{gather}
полученным из (5) при $\xi=0$ и $\eta=y_{0}$,
дает в параметрической форме уравнени
поверхности (рис. 211), отделяющей в пространстве параметров
область автоколебаний от области абсолюной устойчивости.
Точкам ниже поверхности соответствует область
автоколебаний. Точкам выше поверхности - устойчивость
в большом (рис. 212, \textit{а}). Точкам по поверхности - вырожденный

(РИСУНКИ 212)

двойной цикл, проходящий через концы отрезка покоя
(рис. 212, \textit{б}). На рис. 212, \textit{в} изображены два склеенных предельных
цикла - устойчивый и неустойчивый (неустойчивый обозначен
штриховой линией).

Если $r$ "велико" ($y_{0}>1-r$), фазовые траектории подходят с 
обеих сторон к линиям сшивания $y=\pm y_{0}$и система (2) должна
быть из физических соображений доопределена условием
\begin{center}
$$
\dot{x}=y, y=
\left\{
    \begin{array}{ll}
        &y_{0}  \mbox{ при } x \leq -(z-k)/2,\\
        -&y_{0}  \mbox{ при } x \geq (z-k)/2,
    \end{array}
\right.
$$
\end{center}
требующим, чтобы движение продолжалось по линии стыков таекторий
(скользящий режим). Уравнение (4) теряет сысл. Любая 
траектория, сшитая из четырех кусков в верхней полуплоскости,
начинающаяся в точке $(-u, 0)$ и заканчивающаяся в точке
$(v, 0)$, содержит кусок прямой $y=y_{0}$, принадлежащий линии
сшивания. В уравнении (5) праметр $\eta$ принимает фиксорованное
значение $y_{0}$. Уравнения (5) и (6) будут в параметрическом
виде (с параметром $\xi$) связывать $v$ и $k$. Уравнение (9) сохраняет

(РИСУНОК 213)

смысл и для случая сколь угодно больших $r$. На рис. 213 изображены
различные возможные тиы разбиения фазовой плоскости
для этого случая. В отличие от случая "малых $r$", здесь устойчивый
предельный цикл будет вырожденным (на него переходят
точки сконтинуума траекторий).

\textbf{\S 3. Электрическая цепь с туннельным диодом.} Рассматривается
 система [28]
\begin{gather}
\dot{x}=y-\phi(x), \dot{y}=\sigma - \lambda x-y, g>0, \lambda>0,
\end{gather}
где $\phi$ - нелинейная  функция, содержащая "падающий" участок.
Система такого вида встречается при рассмотрениисхем на 
туннельных диодах, а также в ряде других вопросов. Аппроксимируем
$\phi(x)$ кусочно-линейной функцией, состоящей из трех линейных
кусков. наклоны $k$ будем считать: падающего участка
$k=-\alpha_{2}<0$, восходящих $k=\alpha_{1}>0$. Фазовое пространство при
такой аппроксимации разбивается на три части, в каждой из которых
система линейна. В областях \RomanNumeralCaps{1} и \RomanNumeralCaps{3} лежат восходящие
ветви характеристики, в области \RomanNumeralCaps{2} - падающий участок (рис. 214).

\textbf{1. Состояния равновесия. Разбиение пространства параметров
по числу и характеру состояний равновесия.} Возможны одно
или три грубых состояния равновесия. В случае одного состояния
равновесия имеем фокус (узел), всегда устойчивый в областях
\RomanNumeralCaps{1} или \RomanNumeralCaps{3} и неустойчивый в области \RomanNumeralCaps{2}, если $\alpha_{2}>1$. В случае
трех состояний равновесия имеем
всегда устойчивые фокусы (узлы)
в областях \RomanNumeralCaps{1} и \RomanNumeralCaps{3} и седло в области
\RomanNumeralCaps{2}. Куски прямых $\sigma = x_{1} \lambda +y_{1}$
и $\sigma = x_{2} \lambda +y_{2}$ ($x_{1}, y_{1}$ и $x_{2},$
$y_{2}$ - координаты угловых точек
характеристики) при $\lambda \leq \alpha_{2} $ образуют
в плоскости $(\lambda, \sigma)$ дискриминантную 
кривую, отделяющую область
трех состояний равновесия
от области одного состояния равновесия. Точкам дискриминантной
кривой соответствует сшитое состояние
равновесия типа седло-фокуса или седло-узла, и уговой 
точке $(\lambda = \alpha_{2})$ - неустойчивый отрезок покоя, совпадающий с падающим
участком арактеристики.

В случа $\alpha_{2}<1$ невозможны замкнутые траектории и возможными
бифуркациями являются только появлиние и исчезновение
состояний равновесия. Все нижеследующие рассмотрения ведутся
для случая $\alpha_{2}>1$ и $(\alpha - 1)^2<4a_{2}$, допускающего разнообразные
бифуркации.

\textbf{2. Бифуркации состояний равновесия.}

2.1 \textit{Устойчивость состояния равновесия на линии сшивания.}
Пусть прямая $\sigma - \lambda x - y = 0$ проходит через угловую точку
$(x_{1}, y_{1})$ характеристики на границе \RomanNumeralCaps{1} и \RomanNumeralCaps{2} областей и пусть
$\lambda > (\alpha_{2} + 1)^2/4>\alpha_{2}$. Тогда область \RomanNumeralCaps{1} заполнена кусками траекторий 
устойчивого фокуса, а область \RomanNumeralCaps{2} - неустойчивого. Вводим на 
линии сшивания областей \RomanNumeralCaps{1} и \RomanNumeralCaps{2} положительные координаты $S_{0}$ 
и $S_{1}$ (а на линии сшивания областей \RomanNumeralCaps{2} и \RomanNumeralCaps{3} - координаты $S_{2}$ и 
$S_{3}$) (см. рис. 214). Преобразования $S_{0}$ -> $S_{1}$ по траекториям области
\RomanNumeralCaps{1} и $S_{1}$ -> $S_{0}$ по траекториям области \RomanNumeralCaps{2} запишутся так:
\begin{gather}
S_{2}=S_{0}exp\{-h_{1}\pi/ \omega_{1}\}, \overline{S_{0}}=S_{1}exp\{-h_{2}\pi/ \omega_{2}\},
\end{gather}
где $\omega_{i}, -h_{i} (i=1, 2)$ - мнимая и действительная части корней
характеристического уравнения соответственно для областей
\RomanNumeralCaps{1} и \RomanNumeralCaps{2}.

Состоянием равновесия будет сшитый цетр $(\overline{S_{0}}=S_{0})$, если 
$h_{1} \omega_{1}^-1 + h_{2}^-1 \omega_{2}^-1=0$ или, в раскрытом виде,
\begin{center}
$\lambda=\lambda^+ \equiv (\alpha_{1} \alpha_{2} + 1)(\alpha_{1} - \alpha_{2} + 2)^-1$.
\end{center}
Фокус на склейке будет устойчив $(\overline{S_{0}}<S_{0})$ при $\lambda>'lambda^+$ и неустойчив
$(\overline{S_{0}}>S_{0})$ при $\lambda<'lambda^+$.

2.2 \textit{Рождение предельного цикла из состояния равновесия типа
фукос при перемещении состояния равновесия через линию 
сшивания.} Докажем, что в областях \RomanNumeralCaps{1} и \RomanNumeralCaps{2} может существовать не более одного предельного цикла. Рассмотрим преобразование
$S_{0} \rightarrow \overline{S_{0}}$ по траекториям областей \RomanNumeralCaps{1} и \RomanNumeralCaps{2}. Для области \RomanNumeralCaps{1} будет
\begin{gather}
S_{0} = \frac{\delta_{0}}{sin \omega_{1} \tau_{1}} \[ \omega_{1} cos \omega_{1} \tau_{1} + h_{1} sin \omega_{1} \tau_{1} - \omega_{1}e^{h_{1} \tau_{1}} \] = \delta_{0} \xi (\tau_{1}), \\
S_{1} = \frac{\delta_{0}}{sin \omega_{1} \tau_{1}} \[ \omega_{1} cos \omega_{1} \tau_{1} - h_{1} sin \omega_{1} \tau_{1} - \omega_{1}e^{-h_{1} \tau_{1}} \] = \delta_{0} \chi (\tau_{1}),
\end{gather}
где \delta_{0} - расстояние от границы раздела областей \RomanNumeralCaps{1} и \RomanNumeralCaps{2} до состояния
равновесия; \chi и \xi - монотонные функции (возрастающие
или убывающие в зависимости от знака \delta_{0}). Преобразование 
по траекториям области \RomanNumeralCaps{2} записывается аналогично.

Вычисление производной функции последования дает
\begin{gather}
d\overline{S_{0}}/dS_{0} = S_{0}\overline{S_{0}}^-1 exp\{-2(h_{1} \tau_{1} + h_{2} \MakeUppercase{\theta)}\}.
\end{gather}
Здесь \tau и \MakeUppercase{\theta} - время движения соответственно по траекториям
областей \RomanNumeralCaps{1} и \RomanNumeralCaps{2}, $h_{1}=(1 + \alpha_{1})/2 > 0, h_{2}=(1 - \alpha_{2})/2 < 0$.

Пусть состояние равновесия лежит в области \RomanNumeralCaps{1}. Тогда для переодического 
решения $(\overline{S}=S_{0})$ с увеличением S_{0} время \tau_{1} убывает
(до значения \pi/\omega_{1}), время \MakeUppercase{\theta} возрастает (до значения \pi/\omega_{2})
и производная (4) растет. Поэтому может существовать не более 
двух точек пересечения функции последования с биссектрисой,
причем неподвижная точка с меньшей координатой должна быть
устойчива, а с большей - неустойчива. Так как, по предположению,
состояние равновесия лежит в области \RomanNumeralCaps{1} и является устойчивым фокусом, который не может охватываться усттойчивым же
циклом, то в областях \RomanNumeralCaps{1} и \RomanNumeralCaps{2} может существовать не более одного,
причем неустойчивого цикла.

Пусть состояние равновесия лежит в области\RomanNumeralCaps{2}. Тогда с ростом
S_{0} время \tau_{1} растет, а \MakeUppercase{\theta} убывает. Аналогично находим, что
в этом случае может существовать не более одного устойчивого 
предельного цикла.

Пусть $\sigma - \lambda x - y = 0$ проходит через верхнюю угловую точку
характеристики. Рассмотрим два случая.

1. $\lambda > \lambda^+$. Сшитый фокус устойчив. траектория, проходящая 
через нижнюю угловую точку, в силу (2) при t \rightarrow \infty накручивается
к состоянию равновесия. Эта траектория остается спиралью
и при малых смещениях прямой $\sigma - \lambda x - y = 0$. Если при
малом смещении состояние равновесия попадает в область \RomanNumeralCaps{2}, то
оно становится неустойчивым и, следовательно, появляется хотя
бы один устойчивый предельный цикл. По сказанному выше этот
цикл единственный. Пусть после смещения состояние равновесия
попадает в область \RomanNumeralCaps{1}. так как в объединении областей \RomanNumeralCaps{1} и \RomanNumeralCaps{2} возможно
существование не более одного цикла и фокус сохраняет
устойчивость, то, следовательно, циклы не возникают.

2. $\lambda < \lambda^+$. Аналогично находим, что если при малом смещении
состояние равновесия попадает в область \RomanNumeralCaps{2}, то циклы не возникают, 
а если в область \RomanNumeralCaps{1}, то появляется неустойчивый цикл.

2.3 \textit{Рождение предельных циклов (простого или двойного) из
границы области, заполненной замкнутыми траекториями.} Рассмотрим
преобразования $\overline{S_{0}}=f(S_{0})$, склеенные из двух кусков:
$\overline{S_{0}}= \phi(S_{0})$ - по траекториям областей \RomanNumeralCaps{1} и \RomanNumeralCaps{2} и $\overline{S_{0}}=\psi(S_{0})$ - по всем
областям. Покажем, что $f(S_{0})$, дифференцируема в точке скейки.
Преобразование S_{0} \rightarrow S_{1}, по траекториям области \RomanNumeralCaps{1} дано в (3).
Преобразования S_{1} \rightarrow S_{2}, S_{2} \rightarrow S_{3} и S_{3} \rightarrow S_{0} записываются аналогично.
Значение d\overline{S_{0}}/dS_{0} для функции $\psi(S_{0})$ дано в (4), а для 
функции $\psi(S_{0})$ будет
\begin{gather}
d\overline{S_{0}}/dS_{0} = S_{0}\overline{S_{0}}^-1 exp \{-2h_{1}(\tau_{1} + \tau_{3}) - 2h_{2}(\tau_{2} + \tau_{4})\}.
\end{gather}
Здесь \tau_{1} и \tau_{3} - время движения по областям \RomanNumeralCaps{1} и \RomanNumeralCaps{3}, \tau_{2} и \tau_{4} - время движения по верзу и низу области \RomanNumeralCaps{2}.

Пусть S_{0}=S_{0}^* - граничное значение, разделяющее интервалы
определения преобразований $\phi(S_{0})$ и $\psi(S_{0})$. Производные для 
\phi и \psi в точке склейки совпадают: при $S = S_{0}^*$ будет \tau_{3} = 0, \MakeUppercase{\theta} = \MakeUppercase{\theta}^*,
\tau_{2} + \tau_{4} = \MakeUppercase{\theta}^*.

Пусть теперь прямая $\sigma - \lambda x - y = 0$ проходит через угловую точку характеристики $x_{1}, y_{1}$ и $\lambda = \lambda^+$. Покажем, что предельных
циклов нет.

Функция последования на плоскости $(S_{0}, \overline{S_{0}})$ склеена из отрезка
биссектрисы $\overline{S_{0} = S_{0} < S_{0}^*}$ и кривой $\overline{S_{0}} = \psi(S_{0})$. Функция
$\overline{S_{0}} = f(S_{0})$ дифференцирума в точке склейки и, следовательно,
при $\lambda = \lambda^+$ будет $d\overline{S_{0}}/dS_{0} = 1$ (из (5) находим также, что
$d^2\overline{S_{0}}/d^2S_{0}<0$). При возрастании S_{0} от значения S_{0}^* показатель
экспоненты в (5) монотонно убывает от нулевого значения в 
точке склейки ($\tau_{1} = const, \tau_{3}$ растет и $h_{1}>0$; \tau_{2} и \tau_{4} убывают
и $h_{2}<0$). Других точек пересечения (или касания) с биссектрисой,
кроме S_{0} = S_{0}^* располагается ниже биссектрисы. Спирали, сшитые
из траекторий в обласях \RomanNumeralCaps{1}, \RomanNumeralCaps{2} и \RomanNumeralCaps{3}, накручиваются на границу
области, заполненной замкнутыми кривыми, сшитыми из траекторий в областях \RomanNumeralCaps{1} и \RomanNumeralCaps{2}.

При малом изменении параметров \sigma и \lambda функция последования
измененной системы лежит в малой окрестности функции
последования исходной систоемы. Если сдвигаться по полупрямой
$L_{1}=0 (L_{1} \equiv \sigma - \lambda x_{1} - y_{1}, \lambda>\alpha_{2})$ от значения $\lambda = \lambda^+$ в сторону уменьшения \lambda, то функцией последования для $S_{0}<S_{0}^*$ бдует прямая,
проходящая через начало координат выше биссектрисы,
и для $S_{0}>S_{0}^*$ кривая $\overline{S_{0}}=\psi(S_{0})$, пересекающая биссектрису
один раз (в точке склейки $d^2\overline{S_{0}}/dS_{0}^2$). Из
границы области, заполненной замкнутыми кривыми, появляется
единственный устойчивый предельный цикл. При последующем
уменьшении с пачальная точка функции последования переме-
щается из начала координат по оси 50 (наименьшее 5, соответ-
ствует траектории, идущей в устойчивый фокус и касающейся
линии сшивания при $о== 0), и функция последования 5% = /(50)
будет пересекать биссектрису дважды (из фокуса при перемеще-
нии его с линии склейки появляется единственный неустойчивый
предельный цикл). Если сдвинуться по полупрямой в сторону
увеличения Л от значения А == \* и затем уменьшить с, то функ-
ция последования будет целиком лежать ниже биссектрисы. Из
непрерывности и дифференцируемости функции последования
следует, что в любой малой полуокрестности точки (М*, с*) (ни-
же полупрямой) существуют ) и с, для которых функция после-
дования касается биссектрисы. На фазовой плоскости этому со-
ответствует появление двойного цикла. Такие точки образуют би-
фуркационную кривую, выходящую из точки (М*, с’) па полу-
прямой А = 0.

Касание невозможно при $, << 5%, так как в объединении об-
ластей Ги ПП может быть не более одного цикла, п поэтому рож-
дение двойного цикла при измепении параметров происходит при
5, = 5 от границы области, заполненной замкнутыми траек-
ториями.

2.4. Рождение предельных циклов из концов отрезка покоя,
Нусть прямая с — №2 — у = 0 и падающий участок характеристи-
ки совпадают (А == оо). Падающий участок характеристики будет
неустойчивым отрезком покоя, а области Г и ПГ в силу условия
(ол — 1)? < 402 (ем. п. 1) будут заполнены траекториями устой-
чивых фокусов. Легко получить явпое выражение для преобра-
зования в себя полупрямой 50:

5о = боехр {— 2йул/о1} + 6 (ео — 1) (1 + ехр {- Вл/о 1).

Здесь 6 — ширина области ГГ. Преобразование имеет одну устой-
чивую неподвижную точку.

Повернем теперь прямую с — 2 —у=0 вокруг какой-либо
точки на падающем участке против часовой стрелки. Отрезок
покоя при этом разрушается и возникают седло в области ЛГ и
устойчивые фокусы в областях Г и 1П. Пусть будет ® = 09 — в,
где в >>0 и мало. Ограничиваясь степенями © не выше первой,
получим угловые коэффициенты сепаратрис: [- 1 + е/(оз — 1) ]
(для о-сепаратрис), [- @э — #/ (оз — 1)] (для о-сепаратрис).

При = оз траектории, выходящие из точки, в которой при
& == 0 возникает седло, накручиваются на предельный цикл, о-се-
паратрисы седла в области ПГ при малых ев >> 0 лежат в малой
8 31 ЭЛЕКТРИЧЕСКАЯ ЦЕПЬ С ТУННЕЛЬНЫМ ДИОДОМ 413

окрестности траекторий, выходящих из той же точки при & == 0,
и, следовательно, о-сепаратрисы также накручиваются на устой-
чивый предельный цикл, охватывающий все состояния равпове-
сия. Ноэтому о-сепаратрисы могут лишь скручиваться с неустой-
чивых циклов, лежащих в областях /—-ПГ и 1-Й, охватываю-
щих устойчивые фокусы, возникающие при повороте прямой со-
ответственно в областях / и 1. Таким образом, при повороте
прямой са — Ах — у == 0 из концов отрезка покоя появляются устой-
чивые фокусы в сопровождении охватывающих их неустойчивых
циклов (фокусы и циклы возникают одновременно). В окрестности
каждого фокуса лежит единственный предельный цикл. Послед-
нее следует из того, что производная функции последования, по-
строенная с использованием траекторий седла в области Г, будет
также даваться выражением (4), с тем лишь отличием, что © воз-
растанием бо будет @ — со.

3. Бифуркации сепаратрисе.

8.1. Расположелие бифуркационной кривой для. петли. сепара-
трисы. Пусть при с= со и фиксированном А ==)* прямая со —
— && — у —0 проходит через верхнюю угловую точку характерис-
тики. Изменим с на величину х (х —=00— с) и покажем, что пет-
ля сепаратрисы за счет изменения с возникнуть не может. Пусть

$, и 5, — отрезки, отсекаемые а- и @-сепаратрисами линейного
() р р

седла в области ИГ на границе областей Г и ПТ, а бо и 5, — коор-
дннаты по преобразованию (3) на той же границе. Из (3) следует

5, = 6ох [67' (55/60) ], (6)

где 67! — функция, обратная ©. Величины В, и 1, а следователь-
но, и функции % и 5 от с не зависят.
Так как характеристика ‚есть функция кусочно-линейная, то

при изменении с величины $, 51 и бо будут пропорциональны х:

КА = ух, 6 = ук, (7)
5, == },х. (8)
Сшивая проектор на грапице областей Г и П (полагая
5 = $), из и (7) находим
8, = ку [67 (о/р) ] == эх, (9)

а из (8) и (9) —
5/5, = Уз/У» = соп5!.

,
Таким образом, при фиксированном ) величины 5; и 5, на-
ходятся в постоянном отношении и петля сепаратрисы (5, = 51)
за счет изменения со возникнуть не может.
Если прямая с — М; — у =>0 проходит через середину падаю-
щего участка и А == №, таково, что существует петля сепаратрисы
414 МЕТОДЫ ТЕОРИИ БИФУРКАЦИИ В СШИТЫХ СИСТЕМАХ [га. 19

сверху, то в силу симметрии фазового пространства одновременно
должна существовать и петля сепаратрисы снизу. При этом осу-
ществляется условие 3/12 =1. Так как 1в и '› от с не вависят, то
это условие и, следовательно, обе петли сохраняются при } =,
для всех значений о внутри дискриминантной кривой.

3.2. Устойчивость петель сепаратрис. Устойчивость петель со-
паратрис будет определяться знаком седловой величины, если
седло располагается внутри или на границе области /Г (теоре-
мы 44 и 47 в [13] переносятся на случай, когда сшитая петля со-
держит аналитическое седло). В рассматриваемом случае со >> 4
седловая величина положительна (Р; + О, = о, — 1) и петли се-
паратрис пзнутри и снаружи неустойчивы. При изменении па-
раметров к петле стягивается или от нее рождается единственный
неустойчивый предельпый цикл (см. гл. 10, $ 2, 1М, и гл. 17,
$ 4, п. 4)

4. Качественные структуры разбиения фазового пространства.

4.1. Фазовые портреты, соответствующие значениям пара-
метров со’, & и а”, А таким, что прямые с’ М2 — у=0 и а” —
— №2 — у = 0 располагаются симметрично относительно середипы
падающего участка характеристики, будут симметричны относи-
тельно последней. При изучепии разбиения пространства парамет-
ров поэтому можно рассматривать только часть пространства
(А, с) выше либо ниже линии симметрии а — Мо — ус == 0, где хо,
ус — координаты середины падающего участка.

4.2. Рассмотрим структуры разбиспия фазового пространства
и последовательность бифуркаций, переводящих одну структуру в
другую для значений параметров вдоль бифуркационной прямой
с— Ал: — у: == 0 (ли, у: — координаты верхней угловой точки ха-
рактеристики).

Пусть > \* (рис. 215, а). Состояние равновесия -— устойчи-
вый фокус на склейке, и все траектории идут к пему. При 2 ==
==)* (рис. 215,6) возникает область, заполненная замкнутыми
траекториями. Все сшитые по областям /—-1П траектории пакру-
чинаются на границу этой области. При сз < А < &* (рис. 215, в)
фокус на склейке неустойчив и при уменьшении > от значения
№ = \* от границы области, заполненной замкпутыми траектория-
ми, рождается устойчивый предельный цикл. При 3 =»
(рис. 215, г) (острие дискриминантной кривой) падающий участок
характеристики и прямая о — М2 — у =0 совнадают. Возникает
неустойчивый отрезок покоя внутри устойчивого предельного
цикла, При дальнейшем уменьшении 2), вдоль дискриминантной
кривой появляются два состояния равновесия: склеенный вырож-
денный седло-узел (см. гл. 4, $ 2) и устойчивый фокус в области
1. От конца отрезка покоя вместе с фокусом рождается не-
устойчивый предельный цикл (о-сепаратриса вырождепного со-
стояния равновесия идет к устойчивому циклу, охватывающему
все состояния равновесия, о-сепаратриса скручивается © неустой+
$ 31 ЭЛЕКТРИЧЕСКАЯ ЦЕПЬ С ТУННЕЛЬНЫМ ДИОДОМ 415

чивого цикла, охватывающего устойчивый фокус (рис. 215, д)).
Так как со-сепаратриса при = 0 (прямая у ==0) идет в устойчи-
вый узел в области Ш, состояние равповесия в области ПГ пра

\

47

=

|
|

\
\\

/

Рис. 215

изменении параметров вдоль дискриминантной кривой устойчи-
вости не меняет и бесконечность остается неустойчивой, то ис-
чезновение предельных циклов на интервале (<< Л < со может
произойти только за счет слияния предельных циклов с после-
дующим уничтожением двойного цикла. Это может осуществить-
416 МЕТОДЫ ТЕОРИИ БИФУРКАЦИИ В СШИТЫХ СИСТЕМАХ [Гл. 19

ся лишь при посредстве промежуточной бифуркации — появлении
при %==М < со (рис. 215, е) потли сепаратрисы, возникшей из
с- и е-сепаратрис сшитого вырожденного состояния равновесия.

Петля сепаратрисы как снаружи, так и изнутри неустойчива.
Такую петлю можно рассматривать как особый предельный цикл
с состоянием равновесия на нем, отделяющий структуры с неус-
тойчивым предельпым циклом, охватывающим состояние равнове-
сия в области 1, от структур с неустойчивым циклом, охваты-
вающим все состояния равновесия.

При убывании Л до значения А == М. в петлю «влипает» изпут-
ри неусточнвый предельный цикл (рис. 215, е), а при дальней-
шем убывапии } и разрушении петли от нее рождается неустой-
чивый предельный цикл (рис. 215, ж), охватывающий все состоя-
ния равновесия (а-сепаратриса идет в устойчивый фокус в обла-
сти /И, е-сепаратриса скручивается с неустойчивого предельного
цикла, который охватывает оба состояния равнонесия, и между
циклами нет состояний равновесия). При некотором Л =)» < Л,
(рис. 215, з) необходимо возникает полуустойчивый двойной пре-
дельный цикл, исчезающий при убывании 2}. При дальнейшем
убывании № фокусы превратятся в узлы и возпикнет структура,
качественно эквивалентная структуре при ^=0 (рис. 215, и).
(При убывании до значения (1 — с1)?/4 сохрапяется фокус, при
дальнейшем убывании } фокус превращается в узел.)
\end{document}