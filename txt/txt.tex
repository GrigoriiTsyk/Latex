\documentclass{article}
\usepackage[utf8]{inputenc}
\usepackage[russian]{babel}
\usepackage[utf8]{inputenc}
\usepackage[T2A]{fontenc}
\usepackage[russian]{babel}
\usepackage{indentfirst}
\usepackage{graphicx}
\usepackage{hyperref}
\usepackage{etoolbox}
\usepackage{float}
\usepackage{wrapfig}
\usepackage{multicol}
\usepackage{multirow,tabularx}
\usepackage{listings}
\usepackage[table,dvipsnames]{xcolor}
\usepackage{soul}
\usepackage{amsmath,amsfonts,amssymb,amsthm}
\usepackage{fp}
\usepackage{array}
\usepackage{adjustbox}
\usepackage{geometry}
\setlength{\parindent}{1.25cm}

\begin{document}
\textbf{\S 2. Следящая система с люфтом.} Рассмотрим простейшую
следящую систему с люфтом в контактном устройстве и в зуб
чатом зацеплении, описываемую безразмерным уравнением [152]
\begin{gather}
\ddot{x}+\dot{x}=S(x, \dot{x})
\end{gather}
где $x$ - координата сервомотора и $S(x, \dot{x})$ - кусочно-постоянная
(характеризующая безразмерную э. д. с. и сухое трение в системе). 
Общеизвестным приемом при исследовании точечных преобразований 
является представление и исследование точечного преобразования в 
параметрической форме, где в качестве параметра вводится время пробега 
изображающей точки по траекториям системы между точками сшивания.%abzac

Особенностью рассматриваемой задачи является возможность
другого эффективного параметрического представления точечного
преобразования с введением в качестве параметров некоторых
отрезков в фазовом пространстве. Этот прием имеет значение, вы
ходящее за рамки рассматриваемой задачи.%abzac

Разбиение плоскости $(x, \dot{x})$ на области, где $C(x, \dot{x})$ сохраняет
постоянное значение, производится в зависимости от двух пара-
метров $k$ и $z$, характеризующих соответственно люфт в котакт
ном усростве и люфт в зацеплении.%abzac

Запишем уравнение (1) в виде системы
\begin{gather}
\dot{x}=y, \dot{y}=S(x,y)-y
\end{gather} 
и будем рассматривать фазовые траектории на плоскости $(x, y)$.
Разбиение фазовой плоскости на траектории будет симметрично
относительно начала координат, если за начало отсчета принять
середину максимального интерва
ла длиной $z+k$, который сервомо
тор может пройти по инерции. На
рис. 210 изображено разбиение
плоскости $(x, y)$ на десять об
ластей, где $S(x, y)$ сохраняет
постоянные значения, указан
ные на рисунке. Полосы ши
риной $y0$, примыкающие к оси $x$
сверзу или снизу, соответствуют
выбиранию сервомотором люфта
в зубчатом зацеплении, и для
них соответственно $S(x, y)=1$
или $S(x, y)=-1$ (сухим трением
при свободном движении сервомотора пренебрегаем). Полосе шириной $k$, содержащей внутри ось $y$, соответствует выбирание сер
вомотором совместно со следящей осью люфта в коптактном
устройстве при движении по инерции. Здесь $S(x, y)=-r$ или
$S(x, y)=r$ характеризует твердое трение в системе. На других
участках фазовой плоскости величина $S(x, y)$ имеет значение
$\pm 1 \pm r$, где знаки выбираются в зависимости от знака скорости
и знака включенной э. д. с. или 0, если люфт в зацеплении про
ходится по инерции. Величина $y_{0}$-максимаьная скорость, до
которой разгоняется сервомотор, выбирая люфт в зацеплении,
есть однозначная функция параметра $z$ и определяется урав
нением
\begin{gather}
z+y_{0}+ln(1-y_{0})=0.
\end{gather}
Это уравнение получатется, если в (2) положить $S(x,y)=1$ и
потребовать для решения системы (2) выполнения условий $x=$
$=-x_{0}, y=0; x=-x_{0}+z, y=y_{0}$.%abz

Построим точечное преобразование в себя полупрямой $L:$
$y=0, x\leq\frac{-(z+k)}{2}$, примыкающей слева к отрезку покоя: $y
=0, \frac{-(z+k)}{2}<x<\frac{z+k}{2}$. Так как фазовое пространство
симметрично оносительно начала координат, то задача сводится
к построению точечного отображнеия полупрямой $L$ в симметрич
ную полупрямую $L'$, примыкающую к отрезку покоя справа.%абзац

Рассмотрим траекторию в верзней полуплоскости, сшитую из
четырех ксков, начинающуюся в точке $(-u, 0)$ и заканчиваю
щуюся в точке $(v,0)$. ""Сшивание" траекторий в точках разры
ва правых частей системы совершается элементарно, если знак
правой части второго из уравнений (2) не изменяется при пере
ходе через линию сшивания. Так будет, если $y_{0}\leq1-r$, т. е. если
$r$ "не слишком велико". Точки пересечения этой траектории с по
лосой ширины $k$ будут $x=\frac{z-k}{2}, y=\eta$ и $x=\frac{z+k}{2}, y=x_{i}$. Как
оказывается, величины $\eta$ и $\xi$ целесообразно рассматривать как
параметры точечного преобразования.%абзац

Из уравнения (2), полагая $S(x, y)=1$ для первого куска
траектории и $S(x, y)=1-r$ для второго и используя условия
для концов кусков траекторий: $x=-u, y=0; x=-u+z, y=y_{0};$
$x=\frac{z-k}{2} y=r_{i}$, получим
\begin{gather}
u=\frac{z+k}{2} + (1-r)ln\frac{1-r-y_{0}}{1-r-\eta}+y_{0}-\eta, y_{0}\leq\eta<1-r.
\end{gather}
Полагая далее $S(x,y)=-r$ для третьего куска траектории и
$S(x,y)=-1-r$ для четвертого и используя условия для концов
кусков траекторий
\begin{center}
$x=\frac{z-k}{2}, y=\eta; x=\frac{z+k}{2}, y=\xi; x=v, y=0,$
\end{center}
получим
\begin{gather}
r ln(\xi + r) - r ln(\xi+r)+\eta-\xi-k=0,
\end{gather}
\begin{gather}
v=\frac{z+k}{2}+\xi+(1+r)ln\frac{1+r}{\xi+1+r}, 0\leq\xi<\infty. 
\end{gather}
Уравнения (4)-(6) определяют требуемое точечное преобра
зование в параметрической форме с двумя параметрами $\eta$ и $\xi$.
Разбиение фазового пространства $(x,y)$ на траектории определя
ется взаиморасположением кривых $u=u(\eta)$ и $v=v(\eta)$ на сов
мещенных плоскостях $(\nu, u)$ и $(\nu, v)$. Исследование взаиморас
положения кривых проводится элементарно при использовании
$\eta$ и $\xi$ как параметров.%абзац

Из (5) и(6) находим
\begin{center}
$\frac{d\eta}{d\xi}=\frac{\xi(\eta+r)}{\eta(\xi+r)}> 0, \frac{dv}{d\xi}=\frac{\xi}{\xi+1+r}>0.$
\end{center}
Откуда
\begin{gather}
\frac{dv}{d\eta}=\frac{\eta}{1+r+\xi}\frac{\xi+r}{\eta+r}>0.
\end{gather}
Из (4) имеем\\%абзац
\begin{gather}
\frac{du}{d\eta}=\frac{\eta}{1-r-\epsilon}.
\end{gather}
Сравнивая (7) и (8), непосредственно обнаруживаем, что для
любого $\eta$ будет
\begin{center}
$du/d\eta>dv/d\eta$,
\end{center}
\end{document}