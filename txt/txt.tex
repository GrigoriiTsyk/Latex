\documentclass{article}
\usepackage[utf8]{inputenc}
\usepackage[russian]{babel}

\begin{document}
2. Следящая система с люфтом. Рассмотрим простейшую\\
следящую систему с люфтом в контактном устройстве и в зуб-\\
чатом зацеплении, описываемую безразмерным уравнением [152]\\
{\centering$x+x=S(x, x)$(1)}\\
где $x$-координата сервомотора и $S(x, x)$-кусочно-постоянная\\
(характеризующая безразмерную э. д. с. и сухое трение в систе-\\ме).
Общеизвестным приемом при исследовании точечных пре-\\
образований является представление и исследование точечного преобразования в параметрической форме, где в качестве пара-\\
метра вводится время пробега изображающей точки по траекто-\\
риям системы между точками сшивания.\\%abzac
Особенностью рассматриваемой задачи является возможность\\
другого эффективного параметрического представления точечного\\
преобразования с введением в качестве параметров некоторых\\
отрезков в фазовом пространстве. Этот прием имеет значение, вы-\\
ходящее за рамки рассматриваемой задачи.\\%abzac
Разбиение плоскости $(x, x)$ на области, где $C(x, x)$ сохраняет\\
постоянное значение, производится в зависимости от двух пара-\\
метров $k$ и $z$, характеризующих соответственно люфт в котакт-\\
ном усростве и люфт в зацеплении.\\%abzac
Запишем уравнение (1) в виде системы\\
$x=y, y=S(x,y)-y$(2)\\
и будем рассматривать фазовые траектории на плоскости $(x, y)$.\\
Разбиение фазовой плоскости на траектории будет симметрично\\
относительно начала координат, если за начало отсчета принять\\
середину максимального интерва-\\
ла длиной $z+k$, который сервомо-\\
тор может пройти по инерции. На\\
рис. 210 изображено разбиение\\
плоскости $(x, y)$ на десять об-\\
ластей, где $S(x, y)$ сохраняет\\
постоянные значения, указан-\\
ные на рисунке. Полосы ши-\\
риной $y0$, примыкающие к оси $x$\\
сверзу или снизу, соответствуют\\
выбиранию сервомотором люфта\\
в зубчатом зацеплении, и для\\
них соответственно $S(x, y)=1$\\
или $S(x, y)=-1$ (сухим трением\\
при свободном движении сервомотора пренебрегаем). Полосе шириной $k$, содержащей внутри ось $y$, соответствует выбирание сер-\\
вомотором совместно со следящей осью люфта в коптактном\\
устройстве при движении по инерции. Здесь $S(x, y)=-r$ или\\
$S(x, y)=r$ характеризует твердое трение в системе. На других\\
участках фазовой плоскости величина $S(x, y)$ имеет значение\\
$+- 1 +- r$, где знаки выбираются в зависимости от знака скорости\\
и знака включенной э. д. с. или 0, если люфт в зацеплении про-\\
ходится по инерции. Величина $y0$-максимаьная скорость, до\\
которой разгоняется сервомотор, выбирая люфт в зацеплении,-\\
есть однозначная функция параметра $z$ и определяется урав-\\
нением\\
$z+y0+ln(1-y0)=0$ (3)\\
Это уравнение получатется, если в (2) положить $S(x,y)=1$ и\\
потребовать для решения системы (2) выполнения условий $x=$\\
$=-x0, y=0; x=-x0+z, y=y0$.\\%abzac
Построим точечное преобразование в себя полупрямой $L:$\\
$y=0, x<=-(z+k)/2$, примыкающей слева к отрезку покоя: $y=$\\
$=0, -(z+k)/2<x<(z+k)/2$. Так как фазовое пространство\\
симметрично оносительно начала координат, то задача сводится\\
к построению точечного отображнеия полупрямой $L$ в симметрич-\\
ную полупрямую $L'$, примыкающую к отрезку покоя справа.\\%абзац
Рассмотрим траекторию в верзней полуплоскости, сшитую из\\
четырех ксков, начинающуюся в точке $(-u, 0)$ и заканчиваю-\\
щуюся в точке $(v,0)$. ""Сшивание" траекторий в точках разры-\\
ва правых частей системы совершается элементарно, если знак\\
правой части второго из уравнений (2) не изменяется при пере-\\
ходе через линию сшивания. Так будет, если $y0<=1-r$, т. е. если\\
$r$ "не слишком велико". Точки пересечения этой траектории с по-\\
лосой ширины $k$ будут $x=(z-k)/2, y=ny$ и $x=(z+k)/2, y=epc$. Как\\
оказывается, величины $ny$ и $epc$ целесообразно рассматривать как\\
параметры точечного преобразования.\\%абзац
Из уравнения (2), полагая $S(x, y)=1$ для первого куска\\
траектории и $S(x, y)=1-r$ для второго и используя условия\\
для концов кусков траекторий: $x=-u, y=0; x=-u+z, y=y0;$\\
$x=(z-k)/2 y=ri$, получим\\
$u=(z+k)/2 + (1-r)*ln((1-r-y0)/(1-r-ny))+y0-ny, y0<=ny<1-r.$ (4)\\%абзац
Полагая далее $S(x,y)=-r$ для третьего куска траектории и\\
$S(x, y)=-1-r$ для четвертого и используя условия для концов\\
кусков траекторий\\
$x=(z-k)/2, y=ny; x=(z+k)/2, y=\epsilon; x=v, y=0$,\\
получим\\
$r*ln(\epsilon + r) - r*ln(\epsilon+r)+\nu-\epsilon-k=0$, (5)\\
$v=(z+k)/2+\epsilon+(1+r)*ln((1+r)/(\epsilon+1+r), 0<=\epsilon<\infty$. (6)\\%абзац
Уравнения (4)-(6) определяют требуемое точечное преобра-\\
зование в параметрической форме с двумя параметрами $\nu$ и $\epsilon$.\\
Разбиение фазового пространства $(x,y)$ на траектории определя-\\
ется взаиморасположением кривых $u=u(\nu)$ и $v=v(\nu)$ на сов-\\
мещенных плоскостях $(\nu, u)$ и $(\nu, v)$. Исследование взаиморас-\\
положения кривых проводится элементарно при использовании\\
$\nu$ и $\epsilon$ как параметров.\\%абзац
Из (5) и(6) находим\\
$(d\eta)/(d\xi)=(\xi(\eta+r))/(\eta(\xi+r))> 0, (dv)/(d\epsilon)=(\nu/(1+r+\epsilon))*((\epsilon+r)/(\nu+r))>0$. (7)\\%абзац
Из (4) имеем\\%абзац
$du/d\nu=\nu/(1-r-\epsilon)$. (8)\\
Сравнивая (7) и (8), непосредственно обнаруживаем, что для\\
любого $\nu$ будет\\
$du/d\nu>dv/d\nu$,\\
\end{document}